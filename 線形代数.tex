\documentclass[a4j]{jsarticle}

\usepackage{ascmac}
\title{中等教育から見た線形代数学}
\author{田中 拓帆}
\date{2023年8月14日}


\begin{document}
	\maketitle
	\section{はじめに}
	線形代数学は、中等教育課程で用いられるような通常の代数学とは全く異なる学問体系である。しかしながら、、多くの学生諸氏が初めて線形代数に触れる大学数学の解説書等に於いて、線形代数そのものの原初的な説明が著しく不足しているように思う。そこで本書では、線形代数の原初的な概念や考え方から、その原理や性質について、中等教育程度の代数学の知識を用いて理解できるように説明している。
	
	群論等にも触れながら、線形代数とそれに深く関わるその他の学問体系全体を易しく理解できるものとしたい。
	\section{線形とはどういう意味か}
	線形代数は英語ではliniear algebraだ。線形という言葉は、言い換えるとリニア、直線、一次の、という意味があることがわかる。
	
	実際、線形という言葉の定義は、非常にシンプルに直線を言い表している。例えば線形関数の定義は、以下のようになっている。
	\begin{itembox}[l]{線形関数の定義}
		\begin{itemize}
			\item 加法性(additivity) 任意のx,yに対して、 f(x+y)

		\end{itemize}
	\end{itembox}
\end{document}